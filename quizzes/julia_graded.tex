\documentclass[a4paper]{article}
\usepackage[margin=.8in]{geometry}
\usepackage{amsmath}
\usepackage{enumitem}
% \usepackage[cache=true,outputdir=build]{minted}
\usepackage[cache=true]{minted}
\usepackage{amsfonts}
\usepackage{amssymb}
\usepackage{mathtools}
\usepackage{amsmath}
\input{../../macros}

\usepackage{fancyhdr}
\pagestyle{fancy}
\fancyhf{} % clear default
\lhead{Name: } % student writes name
\chead{Quiz 1 -- Introduction to Julia}   % quiz title
\rhead{\today}

% \DeclareUnicodeCharacter{3B8}{θ}
% \DeclareUnicodeCharacter{3B4}{$\bm{\delta}$}
% \DeclareUnicodeCharacter{3BB}{$\lambda$}
% \DeclareUnicodeCharacter{3BC}{$\mu$}
% \DeclareUnicodeCharacter{3BC}{$\pi$}
% \DeclareUnicodeCharacter{3C3}{$\sigma$}
% \DeclareUnicodeCharacter{3C0}{$\pi$}
% \DeclareUnicodeCharacter{3D5}{$\bm{\phi}$}

\begin{document}

\pagestyle{empty}
\thispagestyle{fancy}

% \textbf{Your name:} \\[.5cm]

% \begin{center}
%     \Large{\bf\textsc{Quiz 0 (Introduction to Julia)}}
% \end{center}

True or false? (unless otherwise specified)

\begin{enumerate}

    \item
        In the Julia REPL, the key \julia{;} enables to access \emph{package mode},
        from where software libraries can be installed and uninstalled.

    \item
        In Julia, the first piece of code below (left) produces an error,
        while the second (right) plots the sine function
        (assuming that package \julia{Plots} is already installed).

        \begin{minted}{julia}
    import Pkg                using Plots
    Pkg.add("Plots")          plot(sin)
    plot(cos + sin)               
        \end{minted}

    \item
        Complete the following code so that \julia{f(n)} returns the $n$-th element~$f_n$ of the Fibonacci sequence~$(f_1, f_2, f_3, \dotsc)$. 
        Recall that, by definition, $f_1 = f_2 = 1$ and $f_{n} = f_{n-1} + f_{n-2}$ for $n \geq 3$.
        \begin{minted}{julia}
    f(n) = n in (1, 2) ? 1 : 
        \end{minted}

    \item
        In Julia, the following boolean expression is \julia{true}:
        \begin{minted}{julia}
    sin([1.0, 2.0, 3.0]) == [sin(1.0), sin(2.0), sin(3.0)]
        \end{minted}

    \item
        In the following code, the value of \julia{a} is 15.
        (You may find it useful to read the documentation for the \julia{sum} function to answer this question.)

        \begin{minted}{julia}
    a = sum(n -> n^2 - 1, [1, 2, 3])
        \end{minted}

    \item
        In Julia, the command \julia{[1, 2, 3] .* [1, 2, 3]} returns the array \julia{[1, 4, 9]}.

    \item
        In the following code,
        the variable \julia{a} is an array (more precisely, a \julia{Vector}) containing~$\sqrt{34}$ and~$\sqrt{160}$.
        \begin{minted}{julia}
    hypotenuse(x, y) = sqrt(x^2 + y^2)
    a = hypotenuse.([3, 5], [4, 12])
        \end{minted}

    \item
        The following piece of code prints \julia{2}.

        \begin{minted}{julia}
    f(x::Float64) = 1
    f(x::Int) = 2.0
    @show f(f(1))
        \end{minted}


    \item
        In a Jupyter notebook, the output displayed below a cell comes from the last expression evaluated in that cell. If that expression returns a value 
        (like a number, a string or a plot object), Jupyter automatically displays it.

    \item
        The following program prints the value 15.

        \begin{minted}{julia}
    result = 0
    for i in 1:5
        result = result + i
    end
    println(result)
        \end{minted}

    \item 
        In Julia, arrays are indexed starting from \julia{1}, not \julia{0}.
 
    \item
        In Julia, \julia{=} is the equality operator, while \julia{==} is the assignment operator.

    \item
        In Julia, a function can be defined using the syntax \julia{f(x) = x^2}

    \item
        \textbf{Bonus.} The following code produces an assertion error
        \begin{minted}{julia}
    v = [[1, 2, 3], [2, 3, 4], [4, 5, 6]]
    sum.(v) == [6, 9, 15]
        \end{minted}

    % \item
    %     What is the value of \julia{s} in the following piece of code?
    %     \begin{minted}{julia}
    % struct Wolf end; struct Dog end
    % meet(a::Wolf, b::Wolf) = "Two wolves meet: they howl together."
    % meet(a::Wolf, b::Dog) = "A wolf meets a dog: the wolf growls and the dog is scared."
    % meet(a::Dog, b::Wolf) = meet(b, a)
    % meet(a::Dog, b::Dog) = "Two dogs meet: they wag their tails."
    % raksha, akela = Wolf(), Wolf()
    % s = meet(raksha, akela)
    %     \end{minted}

    % \item
    %     Assume from now on that \julia{A} is a matrix and \julia{b} is a vector,
    %     which were created by the following commands
    %     \begin{minted}{julia}
    % A = randn(7, 7)
    % b = randn(7)
    %     \end{minted}

    %     The command \julia{A[:, 3]} selects the third row of matrix~\julia{A}.

    % \item
    %     The command \julia{A[1:5, :]} returns the submatrix containing the first five \textbf{columns} of~\julia{A}.

    % \item
    %     The command \julia{A[1, :] + b} returns the sum of the first row of \julia{A} and the vector \julia{b}.

    % \item
    %     The command \julia{b[1:end .== 5]} returns the fifth element of \julia{b}.

    % \item
    %     The command \julia{b[b .> 0]} returns a vector containing all and only the positive elements of \julia{b}.
\end{enumerate}

\end{document}
