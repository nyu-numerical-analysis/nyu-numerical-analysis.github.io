\documentclass{article}
\usepackage[margin=1in]{geometry}
\usepackage{amsmath}
\usepackage{enumitem}
\usepackage[cache=true,outputdir=build]{minted}
\usepackage{amsfonts}
\usepackage{mathtools}
\usepackage{amsmath}
\input{../../macros}

\begin{document}
    Throughout the quiz, we consider the following scalar, nonlinear equation,
    where the function $f$ is not necessarily continuous:
    \begin{equation}
        \label{eq:nonlinear}
        f(x) = 0,
        \qquad f\colon \real \to \real,
        \qquad x \in \real.
    \end{equation}

\begin{enumerate}

    \item
        \textbf{(T/F)}
        If~$f$ is strictly increasing,
        then there exists a unique solution to~\eqref{eq:nonlinear}.

        \emph{If you answer false, justify with an counterexample.}
        \vspace{.7cm}

    \item
        \textbf{(T/F)}
        There may exist infinitely many solutions to~\eqref{eq:nonlinear},
        depending on the specific form of~$f$.

        \emph{If you answer true, justify with an example.}
        \vspace{.7cm}

    \item
        \textbf{(T/F)}
        Suppose that $f$ is continuous and that $f(0) f(1) < 0$.
        Then the bisection method, initialized with $a = 0$ and $b = 1$,
        is guaranteed to converge towards a root of~$f$.

    \item
        \textbf{(T/F)}
        Suppose that $f(x) = 3x$,
        and consider the chord method to find a solution to~\eqref{eq:nonlinear}:
        \begin{equation*}
            x_{k+1} = x_k - \frac{f(x_k)}{\alpha}.
        \end{equation*}
        Then, for the function~$f$ given,
        this method converges for~$\alpha = 1$.

    \item
        \textbf{(T/F)}
        Suppose~$f$ is differentiable with a fixed point at~$x_*$,
        and consider the Newton--Raphson method to find a solution to~\eqref{eq:nonlinear}:
        \begin{equation}
            \label{eq:chord}
            x_{k+1} = x_k - \frac{f(x_k)}{f'(x_k)}.
        \end{equation}
        If $f'(x_*) \neq 0$ and $x_k \to x_*$,
        then it holds that
        \[
            \lim_{k \to \infty} \left\lvert \frac{x_{k+1} - x_*}{x_{k} - x_*} \right\rvert
            = 0.
        \]

    \item
        \textbf{(T/F)}
        Sir Isaac Newton was the first person to produce an ultra-efficient implementation of the Newton--Raphson method in \emph{Python},
        which earned him the title of Fellow of the Royal Society.

    \item
        The Newton--Raphson method may be rewritten as a fixed point iteration of the form
        \[
            x_{k+1} = F_{\rm NR}(x_k)
        \]
        for an appropriate function $F_{\rm NR}$.
        Write the expression of the function $F_{\rm NR}$:
        \[
            F_{\rm NR} (x) =
        \]

    \item
        Suppose that $f = x^2$, 
        and that the Newton--Raphson method is employed to find a root of~$f$ starting from~$x_0 = 1$.
        Calculate an explicit expression for $x_k$:
        \[
            x_k = 
        \]

    \item
        \textbf{(2 marks)}
        Write a short Julia code to compute $\sqrt[3]{2}$ to machine precision,
        using the method of your choice and without resorting to the function \julia{cbrt} or \julia{^{1/3}}.
\end{enumerate}
\end{document}



