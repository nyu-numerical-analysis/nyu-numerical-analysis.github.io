\documentclass{article}
\usepackage[margin=1in]{geometry}
\usepackage{amsmath}
\usepackage{enumitem}
% \usepackage[cache=true,outputdir=build]{minted}
\usepackage[cache=true]{minted}
\usepackage{amsfonts}
\usepackage{amssymb}
\usepackage{mathtools}
\usepackage{amsmath}
\input{../../macros}

\usepackage{fancyhdr}
\pagestyle{fancy}
\fancyhf{} % clear default
\lhead{Name: } % student writes name
\chead{Quiz 0 -- Introduction to Julia}   % quiz title
\rhead{\today}

% \DeclareUnicodeCharacter{3B8}{θ}
% \DeclareUnicodeCharacter{3B4}{$\bm{\delta}$}
% \DeclareUnicodeCharacter{3BB}{$\lambda$}
% \DeclareUnicodeCharacter{3BC}{$\mu$}
% \DeclareUnicodeCharacter{3BC}{$\pi$}
% \DeclareUnicodeCharacter{3C3}{$\sigma$}
% \DeclareUnicodeCharacter{3C0}{$\pi$}
% \DeclareUnicodeCharacter{3D5}{$\bm{\phi}$}

\begin{document}

\pagestyle{empty}
\thispagestyle{fancy}

% \textbf{Your name:} \\[.5cm]

% \begin{center}
%     \Large{\bf\textsc{Quiz 0 (Introduction to Julia)}}
% \end{center}

% True or false? (unless otherwise specified)

\begin{enumerate}
    \item
        In the Julia REPL, the key \julia{?} enables to access \emph{package mode},
        from where software libraries can be installed and uninstalled.

    \item
        In the Julia REPL, the key \julia{]} enables to access \emph{help mode},
        from where documentation can be accessed.

    \item
        In Julia, the first piece of code below (left) produces an error,
        while the second (right) plots the sine function
        (assuming that package \julia{Plots} is already installed).

        \begin{minted}{julia}
    import Plots            using Plots
    plot(cos)               plot(sin)
        \end{minted}

    \item
        Explain in words what the following function does? (Assume $n \geq 0$)
        \begin{minted}{julia}
    f(n) = n in (0, 1) ? 1 : n * f(n-1)
        \end{minted}
        \textbf{Explanation:}

    \item
        In Julia, the two following commands produce the same result:
        \begin{minted}{julia}
    v = cos.([1.0, 2.0, 3.0])
    v = [cos(1.0), cos(2.0), cos(3.0)]
        \end{minted}

    \item
        In Julia, the following code plots the function $t \mapsto \cos(t - 1)$
        \begin{minted}{julia}
    import Plots
    shifted_cos(t, p) = cos(t - p)
    Plots.plot(t -> shifted_cos(t, 1))
        \end{minted}

    \item
        In Julia, the command \julia{+(7, 5)} returns 12,
        while \julia{1 .+ [1, 2, 3]} returns the array \julia{[2, 3, 4]}.

    \item
        In the following piece of code,
        the boolean expression on the last line evaluates to \julia{true}:

        \begin{minted}{julia}
    import Base.exp
    exp(a::Vector) = exp(sum(a))
    exp([1, 2, -3]) == 1
        \end{minted}

    \item
        In the following piece of code,
        the boolean expression on the last line evaluates to \julia{false}:

        \begin{minted}{julia}
    import Base.>
    >(a::String, b::String) = length(a) > length(b)
    "Good afternoon" > "world"
        \end{minted}

    \item
        All the code that forms the standard library of the Julia programming language is free
        both as in \emph{free beer} (the price is 0) but also as in \emph{free speech} (you are free to use, modify, and distribute the program).

    \item
        In a Jupyter notebook,
        all the cells are independent.
        In particular, a variable defined in one cell cannot be employed in any of the following cells.

    \item
        In a Jupyter notebook, the output displayed below a cell comes from the last expression evaluated in that cell. If that expression returns a value (like a number, string, DataFrame, or an image/plot object), Jupyter automatically displays it.

    \item
        Loops (\julia{while} and \julia{for}) are much slower in Julia than in Python,
        and so they should be avoided as much as possible.

    \item
        What is the value of \julia{s} in the following piece of code?
        \begin{minted}{julia}
    struct Wolf end; struct Dog end
    meet(a::Wolf, b::Wolf) = "Two wolves meet: they howl together."
    meet(a::Wolf, b::Dog) = "A wolf meets a dog: the wolf growls and the dog is scared."
    meet(a::Dog, b::Wolf) = meet(b, a)
    meet(a::Dog, b::Dog) = "Two dogs meet: they wag their tails."
    raksha, akela = Wolf(), Wolf()
    s = meet(raksha, akela)
        \end{minted}

    % \item
    %     Assume from now on that \julia{A} is a matrix and \julia{b} is a vector,
    %     which were created by the following commands
    %     \begin{minted}{julia}
    % A = randn(7, 7)
    % b = randn(7)
    %     \end{minted}

    %     The command \julia{A[:, 3]} selects the third row of matrix~\julia{A}.

    % \item
    %     The command \julia{A[1:5, :]} returns the submatrix containing the first five \textbf{columns} of~\julia{A}.

    % \item
    %     The command \julia{A[1, :] + b} returns the sum of the first row of \julia{A} and the vector \julia{b}.

    % \item
    %     The command \julia{b[1:end .== 5]} returns the fifth element of \julia{b}.

    % \item
    %     The command \julia{b[b .> 0]} returns a vector containing all and only the positive elements of \julia{b}.
\end{enumerate}

\end{document}
