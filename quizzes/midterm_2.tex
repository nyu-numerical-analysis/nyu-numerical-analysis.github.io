\documentclass{article}
\usepackage[margin=1in]{geometry}
\usepackage{amsmath}
\usepackage{enumitem}
\usepackage[cache=true,outputdir=build]{minted}
\usepackage{amsfonts}
\newcommand{\julia}[1]{\mintinline{julia}{#1}}

\begin{document}
\begin{enumerate}

    \item
        Let $\varepsilon$ denote the machine epsilon for the \julia{Float32} format.
        It holds that $\varepsilon > 10^{-10}$.

    \item
        Machine multiplication is commutative:
        it holds $a \widehat \times b = b \widehat \times a$
        for any \julia{Float64} point numbers $a$ and $b$.

    \item
        If $x \in \mathbb R$ is exactly representable as a \julia{Float32} number,
        then so is $-x$.

    \item
        The only polynomial~$p$ of degree at most 5 such that $p(0) = p(1) = p(2) = p(4) = 0$ is the
        zero polynomial~$p(x) = 0$.

    \item
        Given $x_0 = 0$, $x_1 = 1$, $x_2 = 2$,
        and $y_0, y_1, y_2 \in \mathbb R$,
        the unique quadratic interpolating polynomial through these data points is given by
        \[
            p(x) =
            y_0 + (y_1 - y_0)  x  + \frac{1}{2} \bigl((y_2 - y_1) - (y_1 - y_0)\bigr) x(x-1).
        \]

    \item
        Suppose that $\mathsf A \in \mathbb R^{20 \times 10}$ and $\mathbf b \in \mathbb R^{20}$.
        Then there exists a unique solution to the linear system:
        \[
            \mathsf A^\top \mathsf A \boldsymbol \alpha = \mathsf A^\top \mathbf b.
        \]
    \item
        Let $f(x) = \cos(2x)$,
        and for any $n \in \mathbb N$,
        let $f_n \in \mathcal P_n$ denote the polynomial interpolating~$f$ at $n+1$ equidistant points $-1 = x_0 < x_1 < \dotsc < x_n = 1$.
        Then
        \[
            \lim_{n \to \infty} \left( \max_{x \in [0, \infty)} \bigl\lvert f(x) - f_n(x) \bigr\rvert \right) = 0.
        \]

    \item
        In Julia, if \julia{A} is a matrix,
        then \julia{A[isodd.(1:end), iseven.(1:end)]} returns an empty matrix.

    \item
        The degree of precision of the composite Simpson quadrature rule with $n$ points is equal to $3n$.

    \item
        There exists a unique value of the weights $w_1, w_2, w_3, w_4$ such that the following integration rule has a degree of precision equal to~$3$.
        \[
            \int_{-1}^{1} u(x) \, \mathrm d x 
            = w_1 u(1) 
            + w_2 u \left(\frac{1}{2} \right) 
            + w_3 u \left(\frac{1}{3} \right) 
            + w_4 u \left(\frac{1}{4} \right) 
        \]

\end{enumerate}
\end{document}



