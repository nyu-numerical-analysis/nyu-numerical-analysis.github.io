\documentclass[a4paper]{article}
\usepackage[margin=.8in]{geometry}
\usepackage{amsmath}
\usepackage{enumitem}
\usepackage[cache=true]{minted}
\usepackage{amsfonts}
\usepackage{amssymb}
\usepackage{mathtools}
\usepackage{amsmath}
\input{../../macros}
\usepackage{fancyhdr}
\pagestyle{fancy}
\fancyhf{}
\lhead{Name: }
\chead{Quiz 2 -- Interpolation}
\rhead{\today}

\begin{document}

\pagestyle{empty}
\thispagestyle{fancy}

True or false? (unless otherwise specified)

\begin{enumerate}

    \item
        The only polynomial of any degree such that $p(-1) = 1$, $p(0) = 0$ and $p(1) = 1$ is the quadratic polynomial $p(x) = x^2$.

    \item
        Let $p \in \mathcal P_3$ be given by $p(x) = x^3 + 3x^2 - x + 1$ and let $q \in \mathcal P_2$ denote the polynomial interpolation of~$p$ at $x_0 = 0$, $x_1 = 1$ and $x_2 = 2$.
        Then there exists a constant $C \neq 0$ such that error satisfies
        \[
            \forall x \in \mathbb R, \qquad
            p(x) - q(x) = C (x-x_0) (x-x_1) (x-x_2)
        \]

    \item
        Given $x_0 < x_1 < \dotsc < x_n \in \mathbb R$
        and $y_0, y_1, \dotsc , y_n \in \mathbb R$,
        there exist infinitely many polynomials~$p \in \mathcal P_{n+1}$ such that $p(x_i) = y_i$ for all $i \in \{0, 1, \dotsc, n\}$.

    \item
        In interpolation, the choice of interpolating points can have an influence on the interpolation error.

    \item
        Suppose that $f \colon [-1, 1] \to \mathbb R$ is given by $f(x) = e^x$,
        and for any $n \in \mathbb N$,
        let $f_n \in \mathcal P_n$ denote the polynomial interpolating~$f$ at $n+1$ equidistant points $-1 = x_0 < x_1 < \dotsc < x_n = 1$.
        Then
        \[
            \lim_{n \to \infty} \left( \max_{-1 \leq x \leq 1} \bigl\lvert f(x) - f_n(x) \bigr\rvert \right) = 0.
        \]

    \item
        There exists a polynomial~$p$ such that
        \[
            \forall n \in \mathbb N, \qquad
            p(n) = 2^n \, .
        \]
        \emph{
            Hint: Let $s(n) = 2^n$. If $s$ were a polynomial,
            there would exist $\alpha \in \mathbb N$ such that $\Delta^{\alpha} s = 0$.
        }

    \item
        In Julia, if \julia{A} is a matrix,
        then \julia{A[:, 1:2]} gives the submatrix containing the first two rows of \julia{A}.

    \item
        In Julia, if \julia{A} is a matrix,
        then \julia{A[1:end .!= 2, 1:end .!= 2]} gives the matrix obtained by removing the second column and the second row.

    \item
        In Julia, typing \julia{]} in a REPL (command line) enables to access package mode,
        from which new packages can be installed.

    \item
        In the following code,
        \julia{p} is the interpolating polynomial through the data in \julia{x} and \julia{y}.
        \begin{minted}{julia}
using Plots
x = [0, 1, 2, 3]
y = [1, 2, 1, 2]

function p(x)
    return (y[1]
            + diff(y)[1] * x
            + 1/2 * diff(diff(y))[1] * x * (x-1)
            + 1/6 * diff(diff(diff(y)))[1] * x * (x-1) * (x-2))
end

plot(p, xlims=(0, 5))
scatter!(x, y)
        \end{minted}

    \item \textbf{Bonus}: Obtain an explicit expression for $S(N) := \sum_{n=0}^{N} n^3$ by Gregory--Newton interpolation.
\end{enumerate}

\end{document}
