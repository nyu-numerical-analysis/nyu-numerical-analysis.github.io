\documentclass[a4paper]{article}
\usepackage[margin=.8in]{geometry}
\usepackage{amsmath}
\usepackage{enumitem}
\usepackage[cache=true]{minted}
\usepackage{amsfonts}
\usepackage{amssymb}
\usepackage{mathtools}
\usepackage{amsmath}
\input{../../macros}
\usepackage{fancyhdr}
\pagestyle{fancy}
\fancyhf{}
\lhead{Name: }
\chead{Quiz 6 -- Integration (graded)}
\rhead{\today}
\renewcommand{\d}{\mathrm d}

\begin{document}
\begin{enumerate}

    \item
        The degree of precision of the following quadrature rule is 1:
        \[
            \int_{-1}^{1} f(x) \, \d x
            \approx 2 f(0) - f(-1) + f(1).
        \]

    \item
        The closed Newton--Cotes rule with $n$ points is exact for all polynomials of degree up to $n+1$.

    \item
        Simpson's composite integration rule is always exact if $f$ is a polynomial.

    \item
        The degree of precision of the following quadrature rule is 0:
        \[
            \int_{-1}^1 f(x) \, \d x \approx 2f(1).
        \]

    \item
        Suppose that $f \in C^{\infty}[a, b]$ and let $I_n[f]$ denote the approximate integral of $f$ using the composite Simpson rule
        with~$n$ integration points.
        Then it holds that
        \[
            \lim_{n \to \infty} n^3 \Bigl\lvert I[f] - I_n[f] \Bigr\rvert = 0, \qquad I[f] := \int_{a}^{b} f(x) \, \d x.
        \]

    \item
        Suppose that $f \in C^{\infty}[a, b]$ and let $I_n[f]$ denote the approximate integral of $f$ using the composite trapezium rule
        with~$n$ integration points.
        Then for all $C \in \real$ it holds that
        \[
             \forall n \in \{2, 3, \dotsc \}, \qquad
             \Bigl\lvert I[f] - I_n[f] \Bigr\rvert \leq \frac{C}{n^2}.
        \]

    \item
        Suppose that $f \in C^{\infty}[a, b]$ and let $I_n[f]$ denote the approximate integral of $f$ using the composite trapezium rule
        with~$n$ integration points.
        Then it holds that
        \[
             \lim_{n \to \infty} n \Bigl\lvert I[f] - I_n[f] \Bigr\rvert = 0.
        \]

    \item
        There exist $w_1$ and $w_2$ such that the degree of precision of the following rule is $4$:
        \[
            \int_{-1}^1 f(x) \, \d x
            = w_1 f(0) + w_2 f(1) + w_3 f(-1) + w_4 f \left( \frac{1}{3} \right).
        \]

    \item
        In Julia, the following code implements the composite trapezium rule with $n+1$ points:
        \begin{minted}{julia}
    f(x) = sin(x)
    function I_approx(a, b, n)
        x = LinRange(a, b, n + 1)
        h = x[2] - x[1]
        return h/2 * sum(f, x[1:n]) + h/2 * sum(f, x[2:end])
    end
        \end{minted}

    \item
        In Julia, the following code implements the composite Simpson rule with $n+1$ points:
        \begin{minted}{julia}
    f(x) = sin(x)
    function I_approx(a, b, n)
        x = LinRange(a, b, n + 1)
        h = x[2] - x[1]
        w = [1; [2 + 7(i%2) for i in 1:n-1]; 1]
        return h/3 * w'f.(x)
    end
        \end{minted}

        \item \textbf{Bonus 1.} There exist integration points $x_1, x_2, x_3$ and weights $w_1, w_2, w_3$ such that the following integration rule has a degree of precision equal to 5.
            \[
                \int_{-1}^{1} u(x) \, \d x \approx w_1 u(x_1) + w_2 u(x_2) + w_3 u(x_3) \, .
            \]

        \item \textbf{Bonus 2.} Assume that $x_1 < x_2 < x_3$ are given points. 
            Write expressions for the weights $w_1, w_2, w_3$  to ensure that the following integration rule has a degree of precision equal to at least 2.
            \[
                \int_{-1}^{1} u(x) \, \d x \approx w_1 u(x_1) + w_2 u(x_2) + w_3 u(x_3) \, .
            \]
            You may use the Lagrange polynomials $\ell_1, \ell_2, \ell_3$ associated with the points in these expressions
    \end{enumerate}
\end{document}



