\documentclass{article}
\usepackage[margin=1in]{geometry}
\usepackage{amsmath}
\usepackage{enumitem}
\usepackage[cache=true,outputdir=build]{minted}
\usepackage{amsfonts}
\newcommand{\julia}[1]{\mintinline{julia}{#1}}
\renewcommand{\d}{\mathrm d}

\begin{document}
\begin{enumerate}

    \item
        The degree of precision of the follwing quadrature rule is 1:
        \[
            \int_{-1}^{1} f(x) \, \d x
            \approx 2 f(0) - f(-1) + f(1).
        \]

    \item
        The closed Newton--Cotes rule with $n$ points is exact for all polynomials of degree up to $n+1$.

    \item
        Simpson's composite integration rule is always correct if $f$ is a polynomial.

    \item
        The degree of precision of the following quadrature rule is 0:
        \[
            \int_{-1}^1 f(x) \, \d x \approx 2f(1).
        \]

    \item
        Suppose that $f \in C^{\infty}[a, b]$ and let $I_n[f]$ denote the approximate integral of $f$ using the composite Simpson rule
        with~$n$ integration points.
        Then it holds that
        \[
            \lim_{n \to \infty} n \Bigl\lvert I[f] - I_n[f] \Bigr\rvert = 0, \qquad I[f] := \int_{a}^{b} f(x) \, \d x.
        \]

    \item
        Suppose that $f \in C^{\infty}[a, b]$ and let $I_n[f]$ denote the approximate integral of $f$ using the composite trapezium rule
        with~$n$ integration points.
        Then there exists $C$ such that
        \[
             \forall n \in \{2, 3, \dotsc \}, \qquad
             \Bigl\lvert I[f] - I_n[f] \Bigr\rvert \leq \frac{C}{n^2}.
        \]

    \item
        Suppose that $f \in C^{\infty}[a, b]$ and let $I_n[f]$ denote the approximate integral of $f$ using the composite trapezium rule
        with~$n$ integration points.
        Then it holds that
        \[
             \lim_{n \to \infty} n \Bigl\lvert I[f] - I_n[f] \Bigr\rvert = 0.
        \]

    \item
        There exist $w_1$ and $w_2$ such that the degree of precision of the following rule is $4$:
        \[
            \int_{-1}^1 f(x) \, \d x
            = w_1 f(0) + w_2 f(1) + w_3 f(-1) + w_4 f \left( \frac{1}{3} \right).
        \]

    \item
        In Julia, the following code implements the composite trapezium rule with $n$ points:
        \begin{minted}{julia}
    f(x) = sin(x)
    function I_approx(a, b, n)
        x = LinRange(a, b, n)
        h = x[2] - x[1]
        return h/2 * sum(f, x[1:n]) + h/2 * sum(f, x[2:end])
    end
        \end{minted}

    \item
        In Julia, the following code implements the composite Simpson rule with $n$ points:
        \begin{minted}{julia}
    f(x) = sin(x)
    function I_approx(a, b, n)
        x = LinRange(a, b, n)
        h = x[2] - x[1]
        w = [1; [2 + 2(i%2) for i in 0:n-1]; 1]
        return h/3 * w'f.(x)
    end
        \end{minted}
    \end{enumerate}
\end{document}



