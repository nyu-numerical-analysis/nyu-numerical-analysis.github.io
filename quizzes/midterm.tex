\documentclass[10pt]{article}
\usepackage{fontawesome}
\usepackage{setspace}
\onehalfspacing
\usepackage[colorlinks=true]{hyperref}
\usepackage[outputdir=build,newfloat]{minted}
\usepackage{mdframed}
\usepackage[margin=1.2in]{geometry}
\usepackage{amsmath,amsthm,amssymb}
\usepackage{mathtools}
\theoremstyle{definition}
\newtheorem{question}{{\normalfont \faGears}~Question}
\newtheorem{compexercise}{{\normalfont \faLaptop}~Computer exercise}
\theoremstyle{remark}
\newtheorem*{protosolution}{Solution}
\usepackage{enumitem}
\usepackage{xcolor}
\setlist[enumerate]{font=\bfseries}
\input{../../macros}


\definecolor{lightgreen}{HTML}{f1faf8}
\newenvironment{solutionframe}
{%
    \begin{mdframed}[
        leftmargin=1cm,
        skipabove=.3cm,
        linecolor=blue,
        backgroundcolor=lightgreen,
        linewidth=0pt,
        innerleftmargin=.5em,
        innerrightmargin=.5em,
        innertopmargin=.3em,
        innerbottommargin=.6em,
    ]
}
{
    \end{mdframed}
}

\newenvironment{solution}
{\pushQED{\qed}\renewcommand{\qedsymbol}{$\triangle$}
\begin{solutionframe}\small \begin{protosolution}}
{\popQED\end{protosolution}\end{solutionframe}}

\begin{document}

\title{Numerical Analysis: Midterm \\
\small{(\textbf{50 marks})}}
\author{Urbain Vaes}
\maketitle

\begin{question}
    [Floating point arithmetic, \textbf{10 marks}]
    True or false?
    \begin{enumerate}
        \item
            Let $(\placeholder)_2$ denote binary representation.
            It holds that
            \(
                (0.1011)_2 + (0.0101)_2 = 1.
            \)

        \item
            Let $(\placeholder)_3$ denote base 3 representation.
            It holds that
            \(
                (1000)_3 \times (0.002)_3 = 2.
            \)

        \item
            A natural number with binary representation $(b_4 b_3 b_2 b_1 b_0)_2$ is even if and only if $b_0 = 0$.

        \item
            In Julia, \julia{Float64(.4) == Float32(.4)} evaluates to \julia{true}.

        \item
            Machine addition~$\madd$ is a commutative operation.
            More precisely, given any two double-precision floating point numbers $x \in \floating_{64}$ and $y \in \floating_{64}$,
            it holds that
            \(
                x \madd y = y \madd x.
            \)

        \item
            Let $\floating_{32}$ and $\floating_{64}$ denote respectively the sets of single and double precision floating point numbers.
            It holds that $\floating_{32} \subset \floating_{64}$.

        \item
            In Julia, \julia{eps(Float16)} returns the smallest strictly positive number that can be represented exactly in the \julia{Float16} format.

        \item
            Let $\floating_{64}$ denote the set of double precision floating point numbers.
            For any $x \in \real$ such that~$x \in \floating_{64}$,
            it holds that $x + 1 \in \floating_{64}$.

        \item
            Let $x \in \real$ and $y \in \real$ be two numbers that are exactly representable in the \julia{Float64} format.
            Then $x \madd y = x + y$: machine addition is exact in this case.

        \item
            It holds that $(0.\overline{2200})_3 = (0.9)_{10}$.
    \end{enumerate}
\end{question}

\newpage
\begin{question}
    [Interpolation and approximation, \textbf{10 marks}]
    Are the following assertions true or false?
    Throughout this exercise, we use the notation
    $x^n_i = i/n$.
    The notation~$\poly(n)$ denotes the set of polynomials of degree less than or equal to~$n$.
    We proved in class that,
    for any function $u \colon \real \to \real$ and for all $n \in \nat_{>0}$,
    there exists a unique polynomial~$p_n \in \poly(n)$ such that
    \begin{equation}
        \label{eq:interpolation}
        \forall i \in \{0, \dotsc, n\}, \qquad
        p_n(x^n_i) = u (x^n_i).
    \end{equation}

    \begin{enumerate}
        \item
            If $u$ is not the zero function,
            then the degree of $p_n$ is exactly~$n$.

        \item
            If $u(x) = 2x + 1$,
            then $p_n = u$ for all $n \in \{1, 2, 3, \dotsc\}$.

        \item
            Fix $u(x) = 1 + \sin(57\pi x)$. Then $p_3(x) = 1$.

        \item
            Fix $u(x) = (2x - 1)^3$.
            Then $p_2(x) = 2x - 1$.

        \item
            Fix $n \in \nat_{>0}$ and suppose that $u\colon \real \to \real$ is a smooth function.
            There exists a constant~$K > 0$ independent of~$x$ such that
            \[
                \forall x \in \real,
                \qquad u(x) - p_n(x) = K \prod_{i=0}^{n} \left(x - x^n_i\right).
            \]

        \item
            It holds that
            \[
                \forall x \in [0, 1], \qquad
                \Bigl\lvert (x - x^n_0) \dotsc (x - x^n_n) \Bigr\rvert \leq \left(\frac{1}{n}\right)^n.
            \]

        \item
            In Julia, assuming $n$ and the function $u$ have already been defined,
            the following code enables to calculate the interpolating polynomial~$p_n$ of $u$:
            \begin{minted}{julia}
    using Polynomials
    # Assume `n=5` and `u` have already been defined
    x = LinRange(0, 1, n + 1)
    p = fit(x, u)
            \end{minted}

        \item
            Let $\Delta$ denote the finite difference operator:
            for a function $f \colon \real \to \real$,
            the function $\Delta f$ is defined as
            \[
                \Delta f(x) = f(x + 1) - f(x).
            \]
            Then $f \in \poly(n)$ if and only if $\Delta^{n+1} f = 0$.
            Here $\Delta^{n+1}$ denotes the composition of $n+1$ applications of the operator~$\Delta$.

        \item
            In Julia, the following code enables to fit the data \julia{x} and \julia{y} by a straight line.
            \begin{minted}{julia}
    using Polynomials
    x = [1, 2, 3, 4]
    y = [4, 3, 2, 1]
    p = fit(x, y, 1)
            \end{minted}

        \item
            The choice of interpolation nodes may have an impact on the quality of the interpolation.
\end{enumerate}
\end{question}

\newpage
\begin{question}
    [Numerical integration, \textbf{10 marks}]
    The Gauss--Legendre quadrature formula with $n$ nodes is an approximate integration formula of the form
    \begin{equation}
        \label{gauss_legendre}
        I(u) := \int_{-1}^{1} u(x) \, \d x \approx \sum_{i=1}^{n} w_i \, u(x_i) =: \widehat I_n(u),
    \end{equation}
    which is exact when $u$ is a polynomial of degree less than or equal to $2n-1$.
    (Note that the nodes are here numbered starting from 1.)

    \begin{enumerate}
        \item
            \mymarks{5}
            Find the nodes and weights of the Gauss--Legendre rule with $n=3$ nodes,
            without using any formula (unless you prove it beforehand).

            \textbf{Hint:}
            Recall that a necessary and sufficient condition in order for~\eqref{gauss_legendre} to be satisfied for any polynomial~$p \in \poly(5)$ is that
            \[
                \int_{-1}^{1} x^d \, \d x= \sum_{i=1}^{n} w_i x_i^d,
                \qquad \text{ for all $d \in \{0, 1, 2, 3, 4, 5\}$}.
            \]
            Furthermore, given the symmetry of the problem,
            it is reasonable to look for a solution of the following form,
            which enables to reduce the number of unknowns.
            \[
                (x_1, x_2, x_3, w_1, w_2, w_3)
                = (-x, 0, x, w_1, w_2, w_1).
            \]

        \item \mymarks{5} Are the following assertions true of false :
            \begin{itemize}
                \item
                    The degree of precision of the composite trapezium rule is~$2$.

                \item
                    The composite Simpson rule can be used to integrate exactly a quadratic polynomial.

                \item
                    The degree of precision of the following rule is~$1$
            \begin{minted}{julia}
    function my_integrate(f, a, b)
        x = LinRange(a, b, 100)
        h = x[2] - x[1]
        return h * sum(f, x[1:end-1])
    end
            \end{minted}

                \item
                    The degree of precision of the following integration rule is 2:
                    \[
                        \int_{-1}^{1} f(x) \, \d x \approx
                        2 f(0) + \frac{1}{3} f''(0).
                    \]

                \item
                    Suppose that $u \colon \real \to \real$ is a smooth function,
                    and let $\widehat I_n(u)$ denote an approximation of
                    the integral $I(u) := \int_{-1}^{1} u(x) \, \d x$
                    by the composite trapezium approximation with $n$ points.
                    Let
                    \[
                        \widehat J_n(u) = 2 \widehat I_{2n}(u) - \widehat I_n{u}.
                    \]
                    It holds that
                    \[
                        \lim_{n \to \infty} n^2 \left\lvert I(u) - \widehat J_n(u) \right\rvert
                        = 0.
                    \]
            \end{itemize}

    \end{enumerate}
\end{question}

\begin{compexercise}
    [Interpolation, \textbf{10 marks}]
    Consider the following data:
    \begin{table}[h!]
        \centering
        \begin{tabular}{|c|c|}
            \hline
            \textbf{Time (hours)} & \textbf{Temperature (°C)} \\
            \hline
            6  & 10.5 \\
            \hline
            9  & 15.0 \\
            \hline
            12 & 20.2 \\
            \hline
            15 & 25.1 \\
            \hline
            18 & 22.8 \\
            \hline
            21 & 17.4 \\
            \hline
        \end{tabular}
        \caption{Recorded temperatures at different times of the day.}
    \end{table}

    We wish to approximate the temperature as a smooth function of time.
    To this end, calculate the interpolation polynomial, 
    as well as the best quadratic polynomial approximation (in the sense that the sum of square errors is minimized).
    You may use the \julia{Polynomials} library.
    Plot on the same graph:
    \begin{itemize}
        \item The data points using \julia{scatter};
        \item The polynomial $p_{\rm int}$ interpolating the data points;
        \item The quadratic polynomial $p_{\rm app}$ that best approximates the data,
            in the sense of least squares.
    \end{itemize}
\end{compexercise}

\begin{compexercise}
    [Numerical integration, \textbf{10 marks}]
    Boole's integration rule reads
    \[
        \int_{-1}^{1} u(x) \, \d x
        \approx \frac{7}{45} u(-1) + \frac{32}{45} u\left(-\frac{1}{2}\right) + \frac{12}{45} u\left(0\right) + \frac{32}{45} u\left(\frac{1}{2}\right) + \frac{7}{45} u(1).
    \]
    \begin{itemize}
        \item
            Write a function \julia{comp_boole(u, a, b, N)},
            which returns an approximation of the integral
            \[
                I(u) = \int_{a}^{b} u(x) \, \d x
            \]
            obtained by partitioning the integration interval $[a, b]$ into $N$ cells,
            and applying Boole's rule within each cell.

        \item
            Take $u(x) = \cos(x)$, $a = -1$ and $b = 1$.
            Plot the evolution of the error for $N \in \{1, \dotsc, 1000$.

        \item
            Estimate the order of convergence with respect to $N$, i.e.\ find~$\alpha$ such that
            \[
                \lvert \widehat I_{N} - I \rvert \propto C N^{-\alpha},
            \]
            where $I$ denotes the exact value of the integral
            and $\widehat I_{N}$ denotes its approximation.
            In order to find~$\alpha$,
            use the function \julia{fit} from the \julia{Polynomials} package to find a linear approximation
            of the form
            \[
                \log \lvert \widehat I_{N} - I \rvert \approx \log (C) - \alpha \log(N).
            \]
    \end{itemize}
\end{compexercise}

\end{document}
