\documentclass[a4paper]{article}
\usepackage[margin=.8in]{geometry}
\usepackage{amsmath}
\usepackage{enumitem}
\usepackage[cache=true]{minted}
\usepackage{amsfonts}
\usepackage{amssymb}
\usepackage{mathtools}
\usepackage{amsmath}
\input{../../macros}
\usepackage{fancyhdr}
\pagestyle{fancy}
\fancyhf{}
\lhead{Name: }
\chead{Quiz 4 -- Interpolation and Approximation}
\rhead{\today}

\begin{document}
\begin{enumerate}

    \item
        Given real numbers $x_0 < x_1 < x_2 < x_3$ and $y_0, y_1, y_2, y_3$,
        the unique $p \in \mathcal P_3$ such that~$p(x_i) = y_i$ for all~$i \in \{0, \dotsc, n\}$ is given by
        \begin{align*}
            p(x) &= y_0 \frac{(x-x_1)(x-x_2)(x-x_3)}{(x_0-x_1)(x_0-x_2)(x_0-x_3)}
            + y_1 \frac{(x-x_0)(x-x_2)(x-x_3)}{(x_1-x_0)(x_1-x_2)(x_1-x_3)} \\
                 &\qquad
            + y_2 \frac{(x-x_0)(x-x_1)(x-x_3)}{(x_2-x_0)(x_2-x_1)(x_2-x_3)}
            + y_3 \frac{(x-x_0)(x-x_1)(x-x_2)}{(x_3-x_0)(x_3-x_1)(x_3-x_2)}
        \end{align*}

    \item
        Gregory--Newton interpolation is well suited for incremental interpolation.

    \item
        Suppose that $f\colon[-1, 1] \to \real$ is given by $f(x) = (1 + 25x^2)^{-1}$.
        For integers \(n, m \ge 1\), let \(f_{n,m}: [-1,1] \to \mathbb{R}\) denote the piecewise interpolant obtained by dividing \([-1,1]\) into \(n\) equal subintervals and performing an equidistant polynomial interpolation of degree \(m\) on each subinterval. The resulting function \(f_{n,m}\) is thus a piecewise polynomial approximation of \(f\).
        Then for any fixed $m \geq 0$, it holds that
        \[
            \lim_{n \to \infty} \left( \max_{-1 \leq x \leq 1} \bigl\lvert f(x) - f_{n,m}(x) \bigr\rvert \right) = + \infty \, .
        \]

    \item
        Consider the function $F_n \colon \real^n \to \real$ given by
        \[
            F_n(x_1, \dotsc, x_n) = \max_{x\in [-1, 1]} \Bigl\lvert (x - x_1) \dotsc  (x - x_n) \Bigr\rvert
        \]
        Then for all~$n \in \{1, 2, \dotsc\}$, the minimum of $F_n$ over~$\real^n$ is given by $2/n$.

    \item
        Given (not necessarily distinct) real numbers $x_0, \dotsc, x_n$ and $y_0, \dotsc, y_n$,
        there exists a unique polynomial of degree~$n$ such that
        \[
            \forall i \in \{0, \dotsc, n\}, \qquad
            p(x_i) = y_i.
        \]

    \item
        Using any method you like, write in canonical form the polynomial $p$ interpolating
        the points $(0, 1)$, $(1, 4)$, $(2, 15)$, $(3, 40)$.

        \textbf{Answer:}
        \vspace{1.5cm}

    \item
        In Julia, if \julia{v} is a vector of size 5,
        then \julia{v[v .> 0]} returns a vector with all the positive elements of \julia{v}.

    \item
        In Julia, if \julia{v} is a vector of size 3,
        then \julia{v[[1, 1, 1]]} returns the whole vector \julia{v}.

    \item
        In Julia, if \julia{A} is a $10\times10$ matrix,
        then \julia{A[mod.(1:end, 3) .== 2, :]} gives the matrix obtained by keeping only the third and sixth rows.

    \item
        Write the expression of the polynomial $p$ defined in the following Julia code 
        \begin{minted}{julia}
            x = [1, 2, 3, 4]
            p(z) = prod(z .- x[[1, 3, 4]]) / prod(x[2] .- x[[1, 3, 4]])
        \end{minted}

        \textbf{Answer:}
        \vspace{1.5cm}


    \item
        \textbf{(Bonus)}
        Given real numbers $x_0 < \dotsc < x_n$ and $y_0, \dotsc, y_n$,
        find the \emph{constant} polynomial $p(x) = c$ that best approximates the data in the least–squares sense.
        Equivalently, find $c \in \real$ that minimizes
        \[
            \sum_{i=0}^{n} \lvert y_i - c \rvert^2 \, .
        \]

\end{enumerate}
\end{document}
