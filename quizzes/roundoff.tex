\documentclass[a4paper]{article}
\usepackage[margin=.8in]{geometry}
\usepackage{amsmath}
\usepackage{enumitem}
% \usepackage[cache=true,outputdir=build]{minted}
\usepackage[cache=true]{minted}
\usepackage{amsfonts}
\usepackage{amssymb}
\usepackage{mathtools}
\usepackage{amsmath}
\input{../../macros}

\usepackage{fancyhdr}
\pagestyle{fancy}
\fancyhf{} % clear default
\lhead{Name: } % student writes name
\chead{Quiz 2 -- Floating point arithmetic}   % quiz title
\rhead{\today}

% \DeclareUnicodeCharacter{3B8}{θ}
% \DeclareUnicodeCharacter{3B4}{$\bm{\delta}$}
% \DeclareUnicodeCharacter{3BB}{$\lambda$}
% \DeclareUnicodeCharacter{3BC}{$\mu$}
% \DeclareUnicodeCharacter{3BC}{$\pi$}
% \DeclareUnicodeCharacter{3C3}{$\sigma$}
% \DeclareUnicodeCharacter{3C0}{$\pi$}
% \DeclareUnicodeCharacter{3D5}{$\bm{\phi}$}

\begin{document}

\pagestyle{empty}
\thispagestyle{fancy}

% \textbf{Your name:} \\[.5cm]

% \begin{center}
%     \Large{\bf\textsc{Quiz 0 (Introduction to Julia)}}
% \end{center}

True or false? (unless otherwise specified)

\begin{enumerate}

    \item
        Let $(\placeholder)_2$ denote binary representation.
        It holds that
        \[
            (0.1011)_2 + (0.0101)_2 = 1.
        \]

    \item
        Let $(\placeholder)_7$ denote base 7 representation.
        It holds that
        \[
            (1000)_7 \times (2)_7 = (2000)_7.
        \]

    \item
        Let $p \in \nat$.
        The set
        \(
            \bigl\{ (b_0. b_1 b_2 \dots b_{p-1})_2 \colon b_i \in \{0, 1\} \bigr\} \subset \real
        \)
        contains $2^{p}$ distinct real numbers.

    \item
        The machine epsilon of a floating point format is the smallest strictly positive number that can be represented exactly in the format.

    \item
        Let $\varepsilon$ denote the machine epsilon for the \julia{Float64} format.
        Any $x \in \mathbf R$ such that $-\varepsilon < x < \varepsilon$
        cannot be represented in the \julia{Float64} format.

    \item
        Machine multiplication is commutative,
        meaning that $a \,{\widehat *}\, b = b \, {\widehat *}\, a$
        for any \julia{Float64} point numbers~$a$ and~$b$.

    \item
        If \julia{x} is a \julia{Float16} and \julia{y} is a \julia{Float64} number,
        then the result of \julia{x + y} is a \julia{Float64} number.

    \item
        The real number \julia{0.0} can be represented exactly in the \julia{Float32} format.

    \item
        It holds that
        \[
            (0.\overline{011})_2 = \frac{3}{4}.
        \]

    \item
        In Julia, \julia{Float64(x) == Float32(x)} is \julia{true} if \julia{x} is a rational number.

    \item
        The value of the machine epsilon for \julia{Float64} format is the same as for the \julia{Float32} format.

    \item
        The spacing (in absolute value) between successive double-precision (\julia{Float64}) floating point numbers is always equal to the machine epsilon.

    \item
        \textbf{All} the natural numbers can be represented exactly in the double precision floating point format~\julia{Float64}.

    \item
        Machine addition in the \julia{Float64} format is associative but not commutative.

    \item
        In Julia, \julia{Float64(.4) == Float32(.4)} evaluates to \julia{true}.

    \item
        \textbf{(Bonus)}
        In Julia \julia{10 + eps() == 1} evaluates to \julia{true}.

        \emph{Explain briefly why:}
        \vspace{2.5cm}

    \item
        \textbf{(Bonus)}
         In Julia \julia{sqrt(1 + eps()) == 1 + eps()} evaluates to \julia{false}.

        \emph{Explain briefly why:}
         \vspace{2.5cm}
\end{enumerate}

\end{document}
