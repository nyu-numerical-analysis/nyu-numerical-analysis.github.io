\documentclass{article}
\usepackage[margin=1in]{geometry}
\usepackage{amsmath}
\usepackage{enumitem}
\usepackage[cache=true]{minted}
\usepackage{amsfonts}
\newcommand{\julia}[1]{\mintinline{julia}{#1}}

\begin{document}
\begin{enumerate}

    \item
        Suppose that $x_0 < x_1 < \dotsb < x_n$ and $y_0, y_1, \dotsc, y_n$ are given real numbers.
        Then there exists a unique polynomial $p \in \mathcal P_n$ such that $p(x_i) = y_i$ for all $i \in \{0, \dotsc, n\}$.

    \item
        Suppose that $x_0 < x_1 < \dotsb < x_n$ and $y_0, y_1, \dotsc, y_n$ are given real numbers.
        Then there exists a unique polynomial $p \in \mathcal P_n$ such that $p(x_i) = y_i$ for all $i \in \{0, \dotsc, n\}$.

    \item
        Given $x_0 < x_1 < \dotsc < x_n \in \mathbb R$
        and $y_0, y_1, \dotsc , y_n \in \mathbb R$,
        there exist multiple polynomials~$p \in \mathcal P_{n+1}$ such that $p(x_i) = y_i$ for all $i \in \{0, 1, \dotsc, n\}$.

    \item
        In interpolation, the choice of interpolating points can affect the accuracy of the interpolation.

    \item
        Suppose that $f \colon [-1, 1] \to \mathbb R$ is the function which to $x$ associates $e^x$,
        and for any $n \in \mathbb N$,
        let $f_n \in \mathcal P_n$ denote the polynomial interpolating~$f$ at $n+1$ equidistant points $-1 = x_0 < x_1 < \dotsc < x_n = 1$.
        Then
        \[
            \lim_{n \to \infty} \left( \max_{-1 \leq x \leq 1} \bigl\lvert f(x) - f_n(x) \bigr\rvert \right) = 0.
        \]

    \item
        There exists a polynomial~$p$ such that
        \[
            \forall n \in \mathbb N, \qquad
            p(n) = 2^n \, .
        \]

    \item
        Given $x_0 < x_1 < \dotsc < x_n \in \mathbb R$
        and $y_0, y_1, \dotsc , y_n \in \mathbb R$ with $n = 10$,
        there exists a unique affine polynomial $p(x) = ax + b$ that minimizes
        \[
            J(a, b) = \frac{1}{2} \sum_{i=1}^n \bigl\lvert y_i - p(x_i) \bigr\rvert^2.
        \]

    \item
        In Julia, if \julia{A} is a matrix,
        then \julia{A[:, 1:2]} gives the first two rows of \julia{A}.

    \item
        In Julia, if \julia{A} is a matrix,
        then \julia{A[mod.(1:end, 2) .== 0, mod.(1:end, 2) .== 0]} gives the matrix obtained by removing the second column and the second row.

    \item
        In Julia, typing \julia{;} in a REPL (command line) enables to access package mode,
        from which new packages can be installed.

    \item
        In the following code,
        \julia{p} is the interpolating polynomial through the data in \julia{x} and \julia{y}.
        \begin{minted}{julia}
using Plots
x = [0, 1, 2, 3]
y = [1, 2, 1, 2]

function p(x)
    return (y[1]
            + diff(y)[1] * x
            + 1/2 * diff(diff(y))[1] * x * (x-1)
            + 1/6 * diff(diff(diff(y)))[1] * x * (x-1) * (x-2))
end

plot(p, xlims=(0, 5))
scatter!(x, y)
        \end{minted}
\end{enumerate}
\end{document}
