\documentclass[a4paper]{article}
\usepackage[margin=.8in]{geometry}
\usepackage{amsmath}
\usepackage{enumitem}
\usepackage[cache=true]{minted}
\usepackage{amsfonts}
\usepackage{amssymb}
\usepackage{mathtools}
\usepackage{amsmath}
\input{../../macros}
\usepackage{fancyhdr}
\pagestyle{fancy}
\fancyhf{}
\lhead{Name: }
\chead{Quiz 5 -- Integration (ungraded)}
\rhead{\today}
\renewcommand{\d}{\mathrm d}

\begin{document}
\begin{enumerate}

    \item
        The degree of precision of the follwing quadrature rule is 1:
        \[
            \int_{-1}^{1} f(x) \, \d x
            \approx 2 f(0).
        \]

    \item
        The closed Newton--Cotes rule with $n$ points is exact for all polynomials of degree up to $n$.

    \item
        The trapezium approximation is always wrong if $f$ is not a polynomial.
        That is to say, for all non-polynomial function $f$, it holds that
        \[
            \int_{-1}^1 f(x) \, \d x \neq f(-1) + f(1).
        \]

    \item
        The degree of precision of the following quadrature rule is 1:
        \[
            \int_{-1}^1 f(x) \, \d x \approx f(1).
        \]

    \item
        Suppose that $f \in C^{\infty}[a, b]$ and let $I_n[f]$ denote the approximate integral of $f$ using the composite trapezium rule
        with~$n$ integration points.
        Then it holds that
        \[
            \lim_{n \to \infty} \Bigl\lvert I[f] - I_n[f] \Bigr\rvert = 0, \qquad I[f] := \int_{a}^{b} f(x) \, \d x.
        \]

    \item
        Suppose that $f \in C^{\infty}[a, b]$ and let $I_n[f]$ denote the approximate integral of $f$ using the composite trapezium rule
        with~$n$ integration points.
        Then there exists $C$ such that
        \[
             \forall n \in \{2, 3, \dotsc \}, \qquad
             \Bigl\lvert I[f] - I_n[f] \Bigr\rvert \leq \frac{C}{n^2}.
        \]

    \item
        Suppose that $f \in C^{\infty}[a, b]$ and let $I_n[f]$ denote the approximate integral of $f$ using the composite trapezium rule
        with~$n$ integration points.
        Then it holds that
        \[
             \lim_{n \to \infty} n^2 \Bigl\lvert I[f] - I_n[f] \Bigr\rvert = 0.
        \]

    \item
        There exist $w_1$ and $w_2$ such that the degree of precision of the following rule is $2$:
        \[
            \int_{-1}^1 f(x) \, \d x = w_1 f(0) + w_2 f(1).
        \]

    \item
        In Julia, the following code implements the composite trapezium rule with $n+1$ points:
        \begin{minted}{julia}
    function I_approx(a, b, n)
        x = LinRange(a, b, n + 1)
        h = (b - a) / n
        return h*sum(f, x[1:n] .+ h/2)
    end
        \end{minted}

    \item
        In Julia, the following code implements the composite Simpson rule with $n+1$ points:
        \begin{minted}{julia}
    function I_approx(n)
        x = LinRange(a, b, n + 1)
        h = (b - a) / n
        result = 0.
        result += h/3 * f(x[1]) + h/3 * f(x[end])
        result += 4h/3 * sum(f, x[2:2:end-1])
        result += 2h/3 * sum(f, x[3:2:end-2])
        return result
    end
        \end{minted}
    \end{enumerate}
\end{document}



