\documentclass{article}
\usepackage[margin=1in]{geometry}
\usepackage{amsmath}
\usepackage{enumitem}
\usepackage[cache=true,outputdir=build]{minted}
\usepackage{amsfonts}
\usepackage{mathtools}
\usepackage{amsmath}
\input{../../macros}

\begin{document}
    Throughout the quiz, we consider the following scalar, nonlinear equation:
    \begin{equation}
        \label{eq:nonlinear}
        f(x) = 0,
        \qquad f\colon \real \to \real,
        \qquad x \in \real.
    \end{equation}

\begin{enumerate}

    \item
        Regardless of the specific form of~$f$,
        there exists a unique solution to~\eqref{eq:nonlinear}.

    \item
        There may or may not exist a solution to~\eqref{eq:nonlinear},
        depending on the specific form of~$f$.

        \emph{If you answer true, justify with examples.}
        \vspace{1cm}

    \item
        Suppose there exists a solution to~\eqref{eq:nonlinear}.
        Then the solution is unique.

        \emph{If you answer false, justify with an counterexample.}
        \vspace{1cm}

    \item
        Suppose that $f$ is continuous and that $f(0) f(1) < 0$.
        Then there exists a solution $x_* \in (0, 1)$ to~\eqref{eq:nonlinear}.

    \item
        Suppose that $f$ is continuous, that $f(0) f(1) < 0$,
        and that the bisection method is employed with~$a = 0$ and $b = 1$ in order to find a root of $f$.
        Then it takes fewer than 100 iterations to obtain a approximation of the solution with an error smaller that $10^{-10}$.

    \item
        Suppose that we employ the chord method in order to find a solution to~\eqref{eq:nonlinear}:
        \begin{equation}
            \label{eq:chord}
            x_{k+1} = x_k - \frac{f(x_k)}{\alpha}.
        \end{equation}
        Then this method converges for any value of the parameter~$\alpha \in \real \setminus \{0\}$.

    \item
        Suppose that $f(x) = x$ and that~$\alpha = 2$.
        Then the chord method~\eqref{eq:chord} converges.

    \item
        The chord method may be rewritten as a fixed point iteration of the form
        \[
            x_{k+1} = F_{\rm chord}(x_k)
        \]
        for an appropriate function $F_{\rm chord}$.
        Write the expression of the function $F_{\rm chord}$:
        \[
            F_{\rm chord} (x) =
        \]

    \item
        \textbf{(2 marks)}
        Consider a general fixed point iteration of the form
        \[
            x_{k+1} = F(x_k).
        \]
        Suppose that $F(x_*) = 0$
        and that $F$ is Lipschitz continuous with constant $L < 1$:
        \[
            \forall x, y \in \real, \qquad
            \bigl\lvert F(x) - F(y) \bigr\rvert
            \le L |x - y|.
        \]
        Prove that
        \[
            \lvert x_k - x_* \rvert \le
            L^k \lvert x_0 - x_* \rvert.
        \]
\end{enumerate}
\end{document}



