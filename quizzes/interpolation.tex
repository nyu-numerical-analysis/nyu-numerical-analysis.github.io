\documentclass{article}
\usepackage[margin=1in]{geometry}
\usepackage{amsmath}
\usepackage{enumitem}
\usepackage[cache=true,outputdir=build]{minted}
\usepackage{amsfonts}
\newcommand{\julia}[1]{\mintinline{julia}{#1}}

\begin{document}
\begin{enumerate}

    \item
        The only polynomial of any degree such that $p(0) = 0$ and $p(1) = 1$ is the linear polynomial $p(x) = x$.

    \item
        Given $x_0 < x_1 < \dotsc < x_n \in \mathbb R$
        and $y_0, y_1, \dotsc , y_n \in \mathbb R$,
        which of the following assertions may be false?
        \begin{itemize}[label=$\circ$]
            \item
                there exists a polynomial $p \in \mathcal P_{n+1}$ such that $p(x_i) = y_i$ for all $i \in \{0, 1, \dotsc, n\}$.
            \item
                there exists a unique polynomial $p \in \mathcal P_{n+1}$ such that $p(x_i) = y_i$ for all $i \in \{0, 1, \dotsc, n\}$.
            \item
                there exists a polynomial $p \in \mathcal P_{n}$ such that $p(x_i) = y_i$ for all $i \in \{0, 1, \dotsc, n\}$.
            \item
                there exists a unique polynomial $p \in \mathcal P_{n}$ such that $p(x_i) = y_i$ for all $i \in \{0, 1, \dotsc, n\}$.
        \end{itemize}

    \item
        In interpolation, the choice of interpolating points does not affect the accuracy of the interpolation.

    \item
        In polynomial interpolation, using Chebyshev nodes can help reduce the interpolation error compared to using equidistant nodes.

    \item
        Gregory--Newton interpolation is well-suited for incremental interpolation.

    \item
        Suppose that $f \colon [-1, 1] \to \mathbb R$ is a smooth function,
        and for any $n \in \mathbb N$,
        let $f_n \in \mathcal P_n$ denote the polynomial interpolating~$f$ at $n+1$ equidistant points $-1 = x_0 < x_1 < \dotsc < x_n = 1$.
        Then
        \[
            \lim_{n \to \infty} \left( \max_{-1 \leq x \leq 1} \bigl\lvert f(x) - f_n(x) \bigr\rvert \right) = 0.
        \]

    \item
        Let $(f_0, f_1, f_2, f_3, \dotsc) = (1, 1, 2, 3, \dotsc)$ denote the Fibonacci sequence.
        Does there exist a polynomial~$p$ such that
        \[
            \forall n \in \mathbb N, \qquad
            f_n = p(n) \, ?
        \]
        \emph{%
            Hint: If this is true, then there must be $d \in \mathbb N$ such that $(\Delta^d f)_i = 0$ for all $i$,
            because application of the difference operator $\Delta$ to a polynomial decreases its degree by 1.
        }

    \item
        In Julia, if \julia{A} is a matrix, then \julia{A[:, 1]} gives the first column of \julia{A}.

    \item
        In Julia, if \julia{A} is a matrix, then \julia{A[:, 2:end]} gives the full matrix \julia{A}.

    \item
        What is \julia{p} in the following code?
        \begin{itemize}[label=$\circ$]
            \item The interpolating polynomial.
            \item The Lagrange polynomial associated with $x = 4$.
            \item The Lagrange polynomial associated with $x = 8$.
            \item None of the above.
        \end{itemize}
        \begin{minted}{julia}
using Plots
x = [2, 4, 6, 8, 10]
p(z) = prod((z .- x[1:end .!== 4]) ./ (x[4] .- x[1:end .!== 4]))
plot(p, xlims=(0, 12))
scatter!(x, p)
        \end{minted}
\end{enumerate}
\end{document}
